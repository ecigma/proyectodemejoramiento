\documentclass[12pt]{article}
\usepackage{graphicx}
\usepackage[spanish]{babel}
\begin{document}
\title{ \# 1723  Sandro's Book }
\author{Diego Alfonso Prieto Torres - Sebastian Camilo Martinez Reyes}
\maketitle
\tableofcontents
%PRIMERA SECCION
\section{Introducci\'on}
\\Este documento es una guia de soluci\' on dirigida a los estudiantes para el enunciado \#1723 Sandro's Book del juez virtual UVA, se recomienda a los lectores hacer una previa revisi\'on del enunciado del problema.
\section{Definici\'on del Problema}
\subsection{Objetivos}
Los objetivos del programa con respecto al enunciado son:
\begin{itemize}
\item Determinar el hechizo mas poderoso segun la definici\'on de Sandro.
\end{itemize}
\subsection{Precondici\'on}
La entrada del programa es una linea de texto con el hechizo de Sandro.
\subsection{Poscondici\'on}
la salida debe ser una subcadena de la entrada representando el hechizo mas poderoso.
%SEGUNDA SECCION
\section{Modelo de Soluci\'on}
\subsection{Estrategia de Soluci\'on}
Basta con encontrar la letra del alfabeto ingles con mas incidencias en la linea de texto por entrada, ya que seg\'un la definici\'on de 
Sandro una subcadena del hechizo contiene todo el poder del hechizo original, es decir que una subcadena elemental es 
un hechizo mas poderoso que el original( definimos como subcadena elemental aquellas subcadenas de longitud uno es decir caracteres). 
%TERCERA SECCION
\section{Conclusiones}
Este enunciado nos permite adquirir destreza a la hora de manipular cadenas en el lenguaje de programaci\'on C ya que virtualmente las cadenas de texto estan representadas como arreglos de caracteres y nota un problema para los estudiantes el manejo correcto de estas estructuras.
\end{document}