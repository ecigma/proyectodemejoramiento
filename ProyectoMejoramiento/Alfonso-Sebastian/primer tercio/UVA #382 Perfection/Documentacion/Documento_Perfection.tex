\documentclass[12pt]{article}
\usepackage{graphicx}
\usepackage[spanish]{babel}
\begin{document}
\title{ \# 382  Perfection }
\author{Diego Alfonso Prieto Torres - Sebastian Camilo Martinez Reyes}
\maketitle
\tableofcontents
%PRIMERA SECCION
\section{Introducci\'on}
\\Este documento es una guia de soluci\' on dirigida a los estudiantes para el enunciado \#382 Perfection del juez virtual UVA, se recomienda a los lectores hacer una previa revisi\'on del enunciado del problema.
\section{Definici\'on del Problema}
\subsection{Objetivos}
Los objetivos del programa con respecto al enunciado son:
\begin{itemize}
\item Determinar si un numero dado es perfecto, abundante o deficiente.
\end{itemize}
\subsection{Precondici\'on}
La entrada del programa es una lista de numeros enteros donde 0 es el fin de la lista, es decir no debe ser procesado.
\subsection{Poscondici\'on}
la salida debe ser n lineas donde cada linea corresponde al siguiente formato:
\begin{verbatim}
PERFECTION OUTPUT
i-numero  <Clasificacion>
..
...
...
END OF OUTPUT
\end{verbatim}
\\donde i-numero representa al i-esimo numero de la lista recibida por entrada y \verb < Clasificacion\verb > =\{PERFECT,DEFICIENT,ABUNDANT\} 
\subsection{Ejemplo}
\begin{verbatim}
\\Input:
15 28 6 56 60000 22 496 0
\end{verbatim}
\begin{verbatim}
\\Ouput:
PERFECTION OUTPUT
   15  DEFICIENT
   28  PERFECT
    6  PERFECT
   56  ABUNDANT
60000  ABUNDANT
   22  DEFICIENT
  496  PERFECT
END OF OUTPUT
\end{verbatim}
%SEGUNDA SECCION
\section {Definici\'on de conceptos}
\item se define un divisor propio de $n$ como aquellos numeros que dividen a $n$  donde esos numeros estan entre: $1\leq x<n$ .
\item se dice que un numero $n$ es Perfecto si la suma de sus divisores propios es $n$, ejemplo: los divisores propios de 6 son 1,2,3  1+2+3=6.Se dice que un numero es abundante si la suma de sus divisores propios es mayor al numero y deficiente si es menor.
\section{Modelo de Soluci\'on}
\subsection{Estrategia de Soluci\'on}
Definimos el condjunto $D_n$ asi:
\\
\\$D_n=\{X\mid x|n \wedge x<n \mid X\}$
\\
\\definiremos los metodos perfecto,abundante y deficiente de la siguiente manera:
\\
\\Perfecto.n$\equiv (+i\mid i\in D_n:i)=n$
\\Abundante.n$\equiv (+i\mid i\in D_n:i)>n$
\\Imperfecto.n$\equiv (+i\mid i\in D_n:i)<n$
\\
Asi basta con verificar para cada uno de los elementos de la entrada estas expresiones.
%TERCERA SECCION
\newpage
\section{Conclusiones}
Este enunciado es un claro ejemplo de como podemos expresar  de manera practica las definiciones del mundo de las matematicas en soluciones de software o programas. para responder preguntas simples como si un numero es perfecto abundante o deficiente para el caso de nuestro problema a poder responder expresiones mas complejas haciendo uso de los lenguajes de programac\'on.
\end{document}