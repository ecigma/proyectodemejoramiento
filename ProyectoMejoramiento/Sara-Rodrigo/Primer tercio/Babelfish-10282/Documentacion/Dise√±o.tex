\documentclass[12pt]{article} 

\usepackage[latin1]{inputenc}
\usepackage[spanish]{babel}
\usepackage{color}
\usepackage{multicol}
\usepackage{amsmath}
\usepackage{amssymb}
\usepackage{enumerate}
\usepackage{graphics}
\usepackage{graphicx}

\title{BABELFISH}
\author{Sara Chica, Rodrigo Gualtero}
\date{08 de Diciembre, 2012}

\begin{document}
\maketitle
\tableofcontents

\section{Introducci�n}
Este es un problema de la UVA, identificado con el c�digo \textit{10282}, en el cual se desea saber el significado de ciertas palabras de Waterloo.
\\Es decir, si se tiene en el diccionario lo siguiente:
\begin{center}
dog ogday
\end{center}
Entonces se sabeque cuando se tiene en la entrada <<ogday>>, esto quiere decir dog
\section{Definici�n del problema}
Este problema busca saber el significado de ciertas palabras por medio de las palabras que existen en un diccionario.
\subsection{Entrada}
Entra una lista de palabras, m�ximo 100000, con su significado en el otro idioma. 
\\Seguido a ello entra una linea en blanco, seguido de las palabras que se quieren saber.
\subsection{Salida}
Imprime el significado de cada palabra que se quiere saber.
\\Cuando una palabra no existe en el diccionario, debe imprimir "eh".
\section{Modelamiento matem�tico}
Es importante encontrar f�cilmente cada una de las palabras, debido a que son muchas, por lo tanto se debe usar una estructura de datos que facilite esto.
\section{Planteamiento de la Soluci�n}
La mejor estructura de datos que se puede usar es una hashTable, ya que esta permite una busqueda m�s eficientes, funcionando con una clave generada; permitiendo as� un f�cil acceso a los elementos.
\\En este caso la clave es la palabra en el idiomade Waterloo y el elemento es el significado en ingles.
\section{Conclusiones}
\begin{enumerate}
	\item La HashTable permite busquedas m�s eficientes, ya que utiliza una clave generada, y su tiempo de busqueda depende del n�mero de elementos que tenga.
	\item Las tablas hash almacenan la informaci�n en posiciones pseudo-aleatorias, as� que el acceso ordenado a su contenido es bastante lento.
\end{enumerate}\end{document}