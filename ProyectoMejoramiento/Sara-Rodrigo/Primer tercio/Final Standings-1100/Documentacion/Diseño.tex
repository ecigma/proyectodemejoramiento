\documentclass[12pt]{article} 

\usepackage[latin1]{inputenc}
\usepackage[spanish]{babel}
\usepackage{color}
\usepackage{multicol}
\usepackage{amsmath}
\usepackage{amssymb}
\usepackage{enumerate}
\usepackage{graphics}
\usepackage{graphicx}

\title{FINAL STANDINGS}
\author{Sara Chica, Rodrigo Gualtero}
\date{15 de diciembre, 2012}

\begin{document}
\maketitle
\tableofcontents

\section{Introducci�n}
Este es un problema de TIMUS, identificado con el c�digo \textit{1100}, en el cual se desea generar los resultados finales de acuerdo a la cantidad de problemas que resuelvan en una competencia que se obtenian anteriormente por medio de otro software que lo hac�a m�s lento.
\\Como ahora hay m�s equipos se desea realizar un nuevo programa que permita obtenerlos con mayor rapidez.
\\Esto se realiza ordenando los resultados en orden descendente de acuerdo a la cantidad de problemas resueltos.
\\Un ejemplo de como deber�an quedar organizados es el siguiente:
\begin{center}
	\[1\Rightarrow2\]
	\[16\Rightarrow3\]
	\[11\Rightarrow2\]
	\[20\Rightarrow3\]
	\[3\Rightarrow5\]
	\[26\Rightarrow4\]
	\[7\Rightarrow1\]
	\[22\Rightarrow4\]
	Ejemplo 1.1: Resultados Sin ordenar.
\end{center}

\begin{center}
	\[3\Rightarrow5\]
	\[26\Rightarrow4\]
	\[22\Rightarrow4\]
	\[16\Rightarrow3\]
	\[20\Rightarrow3\]
	\[1\Rightarrow2\]
	\[11\Rightarrow2\]
	\[7\Rightarrow1\]
	Ejemplo 1.2: Resultados Ordenados.
\end{center}

\section{Definici�n del problema}
Este problema busca ordenar en forma descendente un grupo de resultados obtenidos en una competencia en la que cada equipo debe resolver problemas y se mide en la cantidad de problemas que resuelven.
\subsection{Entrada}
En la primera linea entra un valor que determina el n�mero equipos.
\\Seguido a esto entran el n�mero asignado a cada uno de los equipos(ID), junto con la cantidad de problemas resueltos(M).
\\Pueden haber a lo mucho 150000 equipos; y cada uno debe resolver entre 1 a 100 problemas.
\subsection{Salida}
Se debe imprimir los resultados dados en la entrada ordenados en forma decendente, de la forma ID M.
\section{Modelamiento matem�tico}
Se sabe que los resultados est�n organizados si:
\[
Ordenado\equiv(\forall i,j | 0<i<j<15000 : M_{i+1} \leq M_{j})
\]
Para ello existen varios algoritmos de ordenamiento, los cuales permiten organizar una lista en una sencuencia dada, de menor a mayor o viceversa.
\\Algunos de ellos son:
\\Ordenamiento por Burbuja (Bubble Sort): Consiste en organizar el vector revisando cada elemento de la lista y compar�ndolo con el siguiente, de esta forma si est�n en el orden equivocado se deben intercambiar. 
\\Ordenamiento por Mezcla (Merge Sort): Consiste en dividir la lista e ir ordenando cada sublista recursivamente, as� cuando cada sublista est� ordenada se mezclan en una sola ya ordenada.
\\Ordenamiento R�pido (Quick Sort): Se basa en dividir y conquistar, as� pues, coge un elemento como pivote y a partir de este resitua los otros elementos. Este proceso se repite recursivamente hasta tener como resultado toda la lista ordenada.
\\Ordenamiento estable (Stable Sort): Consiste en ordenar un vector de acuerdo a uno de los componentes que contega un valor mayor; sin embargo este conserva el orden relativo de los elementos con valores equivalentes.
\section{Planteamiento de la Soluci�n}
Para resolver este problema se utiliza Stable Sort; este ya se encuentra implementado en varios lenguajes de programaci�n.
\\Stable Sort utiliza un boleano que indica la forma en que se desea ordenar (en caso de no tenerlo se usa por default en forma ascendente).
\\En el ejemplo 1 se organiza de acuerdo al segundo elemento.
\section{Conclusiones}
\begin{enumerate}
	\item Conocer los algoritmos de ordenamiento es importante para optimizar algunos otros algoritmos como los de b�squeda y fusi�n; ya que estos pueden requerir listas ordenadas para tener una ejecuci�n r�pida.
	\item Stable Sort es un buen m�todo de ordenamiento cuando se tienen varios componentes en una estructura y se debe ordenar por medio de alguno de ellos.
\end{enumerate}
\end{document}