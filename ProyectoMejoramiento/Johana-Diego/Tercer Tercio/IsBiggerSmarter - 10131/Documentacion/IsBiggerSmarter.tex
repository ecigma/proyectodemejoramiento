\documentclass[11pt]{article}
\usepackage[latin1]{inputenc}
\usepackage[spanish]{babel}
\usepackage{amsmath}
\usepackage{amssymb}
\usepackage{amsfonts}
\usepackage{multicol}
\usepackage{enumerate}
\usepackage{pstricks,pst-grad,pst-plot,pst-coil}
\usepackage{graphicx}
\newpsobject{grilla}{psgrid}{subgriddiv=1,griddots=10,gridlabels=6pt}
\setlength{\textwidth}{18cm}
\setlength{\textheight}{23cm}
\setlength{\footskip}{1.5cm}
\oddsidemargin-1cm
\topmargin-1.5cm
\setlength{\parindent}{0pt}
\date{}
\renewcommand{\columnseprule}{0.05pt}
\begin{document}
	
	\begin{center}
	\textbf{Is Bigger Smarter?}\\
	
	\textbf{Johanna Beltran y Diego Trivi�o}\\
	
	2012
	\end{center}
	
	\vskip8cm
	
	\tableofcontents
	
	\vskip15cm
	
		\section {Introducci�n}
		
			\vskip0.5cm
		
			\textsl \textbf{Is Bigger Smarter}? es un problema de programaci�n din�mina, el cual encontramos en el juez virtual UVA con el n�mero \textbf{10131}. Este documento busca mostrar una de las tantas soluciones desde el enfoque matem�tico, el objetivo es realizar el c�digo con la soluci�n del problema en cualquier lenguaje de programaci�n con la ayuda de este documento.
		
		\vskip3cm
		
		\section {Definici�n del problema}
		
			\vskip0.5cm
			
			Se deben tomar los datos de una colecci�n de elefantes y mostrar el subconjunto mas grande de manera que los pesos est�n aumentando, pero el coeficiente intelectual disminuyendo.
		
			Se deben tener en cuenta las siguientes restricciones para la soluci�n del problema:
			
			\begin{enumerate}
				\item Cada elefante que se ingresa debe tener sus respectivos datos, es decir peso y coeficiente intelectual.
				
				\item Tanto el peso como el coeficiente intelectual se manejaran en un rango entre 1 y 10000.
				
				\item Solo se podran ingresar hasta 1000 elefantes.
				
				\item Dos elefantes pueden tener el mismo peso � el mismo coeficiente intelectual, incluso el mismo peso y coeficiente intelectual.
				
				\item Todas las desigualdades son estrictas en la soluci�n: Los pesos deben ser estrictamente creciente, y el coeficiente intelectual debe ser estrictamente decreciente.
			\end{enumerate}
	 
	 	\vskip3cm
	
				\subsection {Entrada}
				
				\vskip0.5cm
				
					Se ingresa una pareja de n�meros enteros; el primero representa el peso en kilogramos y el segundo representa el coeficiente intelectual del elefante.
					
					\vskip0.5cm
					
					\[
					 	\emph{EJEMPLO}
					\]
					\[
					 		2  \begin{tabular}{ccc}  8  \end {tabular}
					\]
					\[
					 		1  \begin{tabular}{ccc}  9  \end {tabular}
					\]
					\[
					 		3  \begin{tabular}{ccc}  7  \end {tabular}
					\]
				
				\vskip1cm
				
				\subsection {Salida}
				
					\vskip0.5cm
				
					Se imprime la longitud de la secuencia mas grande de elefantes seguido por los elefantes que la conforman.
					
					\vskip0.5cm
					
					\[
					 	\emph{EJEMPLO ANTERIOR}
					\]
					\[
							3
					\]
					\[
							2
					\]
					\[
							1
					\]
					\[
							3
					\]
	
			\vskip1cm
			
		\section {Modelamiento matem�tico}
		
		\vskip0.5cm
		
			Un elefante se modela como una pareja ordenada $\left(P,IQ \right)$ donde P:peso, IQ:coeficiente intelectual y : 
					
					\[
						1 \leq P \leq 10000 \wedge 1 \leq IQ \leq 10000
					\]
					
					Dada una lista de elefantes de la forma: 
					\[
						$e_{i}$ = $\left(P,IQ\right)$ donde $1 \leq i \leq 1000$
					\]
					
					Una entrada esta dada por una sucesi�n de elefantes $e_{1}$,$e_{2}$,$e_{3}$, ... , $e_{n}$.
					
				\vskip3cm
					
		\section {Planteamiento de la soluci�n}
		
		\vskip0.5cm
		
			Se va a ordenar la lista de elefantes ingresada, de manera que el coeficiente intelectual sea descendente; siendo asi se puede encontrar el elefante estrictamente anterior que cumpla con las restricciones, es decir mayor coeficiente intelectual y menor peso, llevandonos a encontrar una peque�a subsecuencia junto con su tama�o, donde el final de esta es el elefante que estamos evaluando. Al encontrar que el elefante anterior ya tiene una subsecuencia formada se suma al tama�o de la subsecuencia que se esta evaluando.
Es importante saber cual es el tama�o y cuales son los elefantes que conforman la secuencia mas larga, por lo tanto utilizamos una matriz en la cual, en la primera fila se encontraran los tama�os de las subsecuencias de cada elefante ingresado y las dem�s filas seran los elefantes anteriores al evaluado. 
		
		
		\vskip2cm
		
			Se genera un vector B de elefantes ordenados por IQ de manera descendente.
		
			\[
				 \vec{B} = \left\langle f1,f2,f3, \ldots fn \right\rangle
			\]
			
			f es un elefante donde:
			
				\begin{tabular}{ccc} $f_{i} = \left( P_{i} , IQ_{i}) , f_{i-1} = \left( P_{i-1}, IQ_{i-1})$ tal que $IQ_{i} < IQ_{i-1}$ \end {tabular}
			
			Principalmente se piensa en guardar los tama�os de las subsecuencias formadas con cada elefante en un vector T:
				\[
				 \vec{T} = \left\langle t1,t2,t3, \ldots tn \right\rangle
				\]
			
			donde la ecuaci�n recurrente del mayor tama�o de secuencia de elefantes es:
		
				\begin{tabular}{ccc}	$t_{i} = 1 + $ MAX $\left\{t \left(j \right) : 1 \leq j < i \wedge f_{j}$.Peso $ < f_{i}$.Peso $\}$ \end {tabular}
	
			Para guardar no solo los tama�os sino tambi�n los elefantes que conforman la secuencia se crea una matriz L:
				
				\[
				L= \left( \begin{array}{rrr}
				$t_{00}$ & $t_{01}$ & $t_{02}$  ...  $t_{0n}$ \\
				$g_{10}$ & $g_{11}$ & $g_{12}$  ...  $g_{1n}$ \\
				$g_{20}$ & $g_{21}$ & $g_{22}$  ...  $g_{2n}$ \\
				$g_{30}$ & $g_{31}$ & $g_{32}$  ...  $g_{3n}$ \\
				.&.&. \\
				.&.&. \\
				.&.&. \\
				$g_{n0}$ & $g_{n1}$ & $g_{n2}$ ... $g_{nn}$\\
				\end{array}
				\right) \]
					
					Donde cada $t_{0i}$ es el tama�o de la subsecuencia m�s grande que termina en el elefante $f_{i}$, y $g_{ij}$  son los elefantes que conforman la subsecuencia.
					
					Al obtener la matriz podemos encontrar el tama�o m�s grande en la primera fila; siendo as�, podemos obtener los elefantes de la subsecuencia en la misma columna del tama�o pero variando la fila hasta encontrar el �ltimo elefante.
		
			\vskip3cm
		
		\section {Conclusiones}
		
		\vskip0.5cm
		
			\begin{enumerate}
			
				\item Para este problema es necesario y conveniente utilizar programaci�n din�mica porque no existe la necesidad de recalcular aquellas soluciones que ya ten�amos en alg�n momento de la ejecuci�n; ahorrando tiempo y codificaci�n en la soluci�n.
				
				\item Para un problema puede existir muchas soluciones; tenga en cuenta que el vector de elefantes inicial puede ser ordenado tanto por coeficiente intelectual como por peso, manteniendo siempre las restricciones de la soluci�n.
			
			\end{enumerate}
	
\end{document}