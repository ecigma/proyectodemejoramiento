\documentclass[11pt]{article}
\usepackage[latin1]{inputenc}
\usepackage[spanish]{babel}
\usepackage{amsmath}
\usepackage{amssymb}
\usepackage{amsfonts}
\usepackage{multicol}
\usepackage{enumerate}
\usepackage{pstricks,pst-grad,pst-plot,pst-coil}
\usepackage{graphicx}
\newpsobject{grilla}{psgrid}{subgriddiv=1,griddots=10,gridlabels=6pt}
\setlength{\textwidth}{18cm}
\setlength{\textheight}{23cm}
\setlength{\footskip}{1.5cm}
\oddsidemargin-1cm
\topmargin-1.5cm
\setlength{\parindent}{0pt}
\date{}
\renewcommand{\columnseprule}{0.05pt}
\begin{document}
	
	\begin{center}
	\textbf{CUTTING STICKS}\\
	
	\textbf{Johanna Beltran y Diego Trivi�o}\\
	
	2012
	\end{center}
	
	\vskip8cm
	
	\tableofcontents
	
	\vskip15cm
	
		\section {Introducci�n}
		
			\vskip0.5cm
		
			\textsl \textbf{Cutting Sticks} es un problema de programaci�n din�mina, el cual encontramos en el juez virtual UVA con el n�mero \textbf{10003}. Este documento busca mostrar una de las tantas soluciones desde el enfoque matem�tico, el objetivo es realizar el c�digo con la soluci�n del problema en cualquier lenguaje de programaci�n con la ayuda de este documento.
		
		\vskip1cm
		
		\section {Definici�n del problema}
		
			\vskip0.5cm
			
			Para este problema se debe averiguar el costo m�nimo para cortar una vara determinada.
		
			Se deben tener en cuenta las siguientes restricciones para la soluci�n del problema:
			
			\begin{enumerate}
				\item Solo se puede realizar un corte a la vez.
				
				\item La longitud L de la vara debe ser: $L \leq 1000$.
				
				\item El numero de cortes N a realizar debe ser: $N \leq 50$.
				
				\item Al ingresar los cortes $C_{i}$ a realizar sobre la vara se debe tener en cuenta que: $0 \leq C_{i} \leq L$ .
				
				\item Los cortes de la colecci�n deben ser ingresados de manera creciente estrictamente.
				
				\item El final del problema se da cuando ya no hayan cortes por realizar.
				
				\item El costo de la vara se da de acuerdo a su longitud.
			\end{enumerate}
	 
	 	\vskip1cm
	
				\subsection {Entrada}
				
				\vskip0.5cm
				
					La entrada constar� de varios casos de entrada. La primera l�nea de cada caso de prueba contiene un n�mero positivo L que representa la longitud de la 						vara a cortar; la siguiente l�nea contendr� el n�mero N de los cortes a realizar y la siguiente l�nea se compone de N n�meros positivos, que representan 					 los lugares donde los cortes tienen que hacerse.
					
					El final de la entrada de los datos se indica con el n�mero 0.
					
					\vskip0.5cm
					
					\[
					 	\emph{EJEMPLO}
					\]
					\[
							100
					\]
					\[
							3
					\]
					\[
							25 \begin{tabular}{ccc} 50 \begin{tabular}{ccc} 75 \end {tabular} \end {tabular}
					\]
					\[
							10
					\]
					\[
							4
					\]
					\[
							4 \begin{tabular}{ccc} 5 \begin{tabular}{ccc} 7 \begin{tabular}{ccc} 8 \end {tabular} \end {tabular} \end {tabular}
					\]
					\[
							0 
					\]
				
				\vskip1cm
				
				\subsection {Salida}
				
					\vskip0.5cm
				
					Se tiene que imprimir el costo de la soluci�n �ptima del problema de corte, que es el costo m�nimo de cortar la vara dada.
					
					\vskip0.5cm
					
					 	\emph{EJEMPLO ANTERIOR}
					 	
					 	\begin{tabular}{ccc}
					\[
							El corte m�nimo es de 200.
					\]
						\end {tabular}
						
						\begin{tabular}{ccc}
					\[
							El corte m�nimo es 22.
					\]
						\end {tabular}
	
			\vskip3cm
			
		\section {Modelamiento matem�tico}
		
		\vskip0.5cm
		
			Dada la longitud de la vara L y el n�mero de cortes a realizar N: 
					
					Se recibe una sucesion de cortes $C_{1}$, $C_{2}$, $C_{3}$, ..., $C_{N}$
					
					Donde $C_{i}$ \leq L
					
				\vskip3cm
					
		\section {Planteamiento de la soluci�n}
		
		Se genera un vector en el cual la primera posici�n siempre sera 0, las posiciones que le siguen son los cortes ingresados y la ultima posici�n sera la longitud total de la vara ingresada; esto se hace con el fin de tener el valor inicial y final de la vara (desde cero hasta el tama�o de la vara) 'en el mismo vector de los cortes'.
		Se deben evaluar cada uno de los costos totales que se pueden dar para una vara, puesto que estos var�an dependiendo del orden en el que se hagan los cortes.
		Es importante saber los distintos costos que se generan, por lo tanto utilizamos una matriz en la cual se encontraran el costo minimo para cada una de las variaciones de cortes. 
		
		\vskip0.5cm
		
		Se genera un vector G:
		
		\[
			\vec{G} = \left\langle g0,g1,g2, ... , gn, gn+1 \right\rangle
		\]
		
		Donde $g_{0} = 0$, $g_{n+1} = L$ y $g_{i} = C_{j}$
		
		La ecuaci�n recurrente del minimo costo para cortar una vara es:
		
			\[
					
					Minimo(i,j)= \begin{cases}
					\ $0$, & \text{si} \hskip0.3cm \neg \exists $ _{k}$ \left| \hskip0.3cm $g_{i} < g_{k} < g_{j}$
					\\Minimo \left(i,k) \hskip0.3cm + \hskip0.3cm Minimo \left(k,j) \hskip0.3cm + \hskip0.3cm $g_{j}$ \hskip0.3cm - \hskip0.3cm $g_{i}$, & \text{si} \hskip0.3cm \exists $ _{k}$ \left| \hskip0.3cm $g_{i} < g_{k} < g_{j}$
					\end{cases}
			\]
			
			Donde  i,j,k son subindices del vector G, en el primer caso i = 0 y j = N+1; ya que inicialmente queremos saber el costo de cortar una vara que mide desde 0 hasta L
				
				Se crea una matriz M:
				
				\[
				M= \left( \begin{array}{rrr}
				$m_{00}$ & $m_{01}$ & $m_{02}$  ...  $m_{0N}$ \\
				$m_{10}$ & $m_{11}$ & $m_{12}$  ...  $m_{1N}$ \\
				$m_{20}$ & $m_{21}$ & $m_{22}$  ...  $m_{2N}$ \\
				$m_{30}$ & $m_{31}$ & $m_{32}$  ...  $m_{3N}$ \\
				.&.&. \\
				.&.&. \\
				.&.&. \\
				$m_{N0}$ & $m_{N1}$ & $m_{N2}$ ... $m_{NN}$\\
				\end{array}
				\right) \]
					
					Donde cada $m_{ij}$ son los diferentes costes calculados.
	
			\vskip3cm
		
		\section {Conclusiones}
		
		\vskip0.5cm
		
			\begin{enumerate}
			
				\item Para este problema es necesario y conveniente utilizar programaci�n din�mica porque no existe la necesidad de recalcular aquellas soluciones que ya ten�amos en alg�n momento de la ejecuci�n; ahorrando tiempo y codificaci�n en la soluci�n.
				
				\item Para cada entrada existe una �nica respuesta; tenga en cuenta que para esta soluci�n al vector de cortes se le suman dos valores m�s que se deben tener en cuenta para su soluci�n que son el 0 y la longitud total de la vara las cuales deben de ir en orden ascendente junto con los cortes dados.
			
			\end{enumerate}
	
\end{document}