\documentclass[11pt]{article}
\usepackage[latin1]{inputenc}
\usepackage[spanish]{babel}
\usepackage{amsmath}
\usepackage{amssymb}
\usepackage{amsfonts}
\usepackage{multicol}
\usepackage{enumerate}
\usepackage{pstricks,pst-grad,pst-plot,pst-coil}
\usepackage{graphicx}
\newpsobject{grilla}{psgrid}{subgriddiv=1,griddots=10,gridlabels=6pt}
\setlength{\textwidth}{18cm}
\setlength{\textheight}{23cm}
\setlength{\footskip}{1.5cm}
\oddsidemargin-1cm
\topmargin-1.5cm
\setlength{\parindent}{0pt}
\date{}
\renewcommand{\columnseprule}{0.05pt}
\begin{document}
	
	\begin{center}
	\textbf{Age Sort}\\
	
	\textbf{Johanna Beltran y Diego Trivi�o}\\
	
	2012
	\end{center}
	
	\vskip8cm
	
	\tableofcontents
	
	\vskip15cm
	
		\section {Introducci�n}
		
			\vskip0.5cm
		
			\textsl '\textbf{Age Sort}' es un problema de programaci�n el cual encontramos en el juez virtual UVA con el n�mero \textbf{11462}. 
			
			Este documento busca mostrar una de las tantas soluciones desde el enfoque matem�tico teniendo en cuenta que el objetivo es realizar la implementaci�n de la soluci�n del problema en cualquier lenguaje de programaci�n con la ayuda de este documento.
			
			Este problema puede ser resuelto utilizando la estrategia de 'divide y vencer�s'.
			
			Esta estrategia es una t�cnica de dise�o de algoritmos la cual consiste en dividir de forma recurrente un problema en subproblemas m�s sencillos hasta que se encuentre un caso base. 
			
			\vskip1cm
			
			Esta t�cnica consta fundamentalmente de los siguientes pasos:
			
			\begin{enumerate}
			
				\item Descomponer el problema a resolver en un cierto n�mero de subproblemas m�s peque�os.
				
				\item Resolver independientemente cada subproblema.
				
				\item Combinar los resultados obtenidos para construir la soluci�n del problema original.
				
			\end{enumerate}
		
		\vskip4cm
		
		\section {Definici�n del problema}
		
			\vskip0.5cm
			
			Este problema implica clasificar todas las edades en orden ascendente.
		
			\vskip0.5cm
		
			Se deben tener en cuenta las siguientes restricciones para la soluci�n del problema:
			
			\begin{enumerate}
			
				\item Para cada caso de prueba se debe ingresar un n�mero $n$ y una secuencia de $n$ n�meros enteros.
				
				\item Para $n$ se debe tener en cuenta que $1 \leq n \leq 2000000$ y que cada n�mero que conforma la secuencia dada debe ser $1 \leq p \leq 100$ .
				
			\end{enumerate}
	 
	 	\vskip7cm
	
				\subsection {Entrada}
				
				\vskip0.5cm
				
					Hay varios casos de prueba en el archivo de entrada. Cada caso comienza con un entero $n$ $ \left(1 \leq n \leq 2000000)$ que indica el n�mero total de personas. En la siguiente l�nea, deben haber $n$ n�meros enteros que indican las edades. La entrada termina cuando $n = 0$, este caso no se debe tener en cuenta.
					
					\vskip0.5cm
					
					\[
					 	\emph{EJEMPLO}
					\]
					\[
					 		5
					\]
					\[
					 		3 \begin{tabular}{ccc} 4 \begin{tabular}{ccc} 2 \begin{tabular}{ccc} 1 \begin{tabular}{ccc} 5 \end {tabular} \end {tabular} \end {tabular} \end {tabular}
					\]
					\[
					 		5
					\]
					\[
					 		2 \begin{tabular}{ccc} 3 \begin{tabular}{ccc} 2 \begin{tabular}{ccc} 3 \begin{tabular}{ccc} 1 \end {tabular} \end {tabular} \end {tabular} \end {tabular}
					\]
					\[
					 		0
					\]
				
				\vskip2cm
				
				\subsection {Salida}
				
					\vskip0.5cm
				
					Para cada caso de prueba se debe imprimir una l�nea con $n$ enteros separados por espacios. Estos enteros son las edades ingresadas en orden ascendente.
					
					\vskip0.5cm
					
					\[
					 	\emph{EJEMPLO ANTERIOR}
					\]
					\[
							1 \begin{tabular}{ccc} 2 \begin{tabular}{ccc} 3 \begin{tabular}{ccc} 4 \begin{tabular}{ccc} 5 \end {tabular} \end {tabular} \end {tabular} \end {tabular}
					\]
					\[
							1 \begin{tabular}{ccc} 2 \begin{tabular}{ccc} 2 \begin{tabular}{ccc} 3 \begin{tabular}{ccc} 3 \end {tabular} \end {tabular} \end {tabular} \end {tabular}
					\]
	
			\vskip2cm
			
		\section {Modelamiento matem�tico}
		
		\vskip0.5cm
			
		Dado un n�mero entero positivo $n$, se recibe una secuencia de $n$ n�meros enteros: 
		
			\begin{tabular}{ccc} \begin{tabular}{ccc} \begin{tabular}{ccc} \begin{tabular}{ccc} \begin{tabular}{ccc} \begin{tabular}{ccc} \begin{tabular}{ccc} \begin{tabular}{ccc} \begin{tabular}{ccc} \begin{tabular}{ccc} \begin{tabular}{ccc} \begin{tabular}{ccc} \begin{tabular}{ccc} \begin{tabular}{ccc} \begin{tabular}{ccc} \begin{tabular}{ccc} \begin{tabular}{ccc} \begin{tabular}{ccc} \begin{tabular}{ccc} \begin{tabular}{ccc} \begin{tabular}{ccc} \begin{tabular}{ccc} \begin{tabular}{ccc} \begin{tabular}{ccc} \begin{tabular}{ccc} \begin{tabular}{ccc} \begin{tabular}{ccc} \begin{tabular}{ccc} \begin{tabular}{ccc} \begin{tabular}{ccc} $E_{1}$, $E_{2}$, $E_{3}$, ..., $E_{n}$ \end {tabular} \end {tabular} \end {tabular} \end {tabular} \end {tabular} \end {tabular} \end {tabular} \end {tabular} \end {tabular} \end {tabular} \end {tabular} \end {tabular} \end {tabular} \end {tabular} \end {tabular} \end {tabular} \end {tabular} \end {tabular} \end {tabular} \end {tabular} \end {tabular} \end {tabular} \end {tabular} \end {tabular} \end {tabular} \end {tabular} \end {tabular} \end {tabular} \end {tabular} \end {tabular}
					
				\vskip4cm
					
		\section {Planteamiento de la soluci�n}
		
		\vskip0.5cm
		
			Para este problema utilizaremos el algoritmo de ordenamiento \textbf{Quicksort} que permite, en promedio, ordenar $n$ elementos en el menor tiempo posible; el algoritmo trabaja de la siguiente forma:
			
				\begin{enumerate}
				\includegraphics[width=0.50\textwidth]{quicksort2.png}
				
				\footnote{http://www.algolist.net/Algorithms/Sorting/Quicksort}
				
				\vskip1cm
				
				\item Elegir un elemento de la lista de elementos a ordenar, al que llamaremos pivote.
				
				\vskip1cm
				
				\includegraphics[width=0.50\textwidth]{quicksort3.png}
				
				\footnote{http://www.algolist.net/Algorithms/Sorting/Quicksort}
				
				\vskip1cm
				
				\item Resituar los dem�s elementos de la lista a cada lado del pivote, de manera que a un lado queden todos los menores que �l, y al otro los mayores. Los elementos iguales al pivote pueden ser colocados tanto a su derecha como a su izquierda, dependiendo de la implementaci�n deseada. En este momento, el pivote ocupa exactamente el lugar que le corresponder� en la lista ordenada.
				
					\footnote{http://www.algolist.net/Algorithms/Sorting/Quicksort}
				\vskip1cm
				
				\includegraphics[width=0.50\textwidth]{quicksort4.png}
				
				\vskip1cm
				
				\includegraphics[width=0.50\textwidth]{quicksort5.png}
				
				\vskip1cm
				
				\includegraphics[width=0.50\textwidth]{quicksort6.png}
				
				\vskip1cm
				
				\includegraphics[width=0.50\textwidth]{quicksort7.png}
				
				\vskip1cm
				
				\item La lista queda separada en dos sublistas, una formada por los elementos a la izquierda del pivote, y otra por los elementos a su derecha.
				\item Repetir este proceso de forma recursiva para cada sublista mientras �stas contengan m�s de un elemento. Una vez terminado este proceso todos los elementos estar�n ordenados. Como se puede suponer, la eficiencia del algoritmo depende de la posici�n en la que termine el pivote elegido.
				
				\vskip1cm
				
				\includegraphics[width=0.50\textwidth]{quicksort8.png}
				
				\vskip1cm
				
				\includegraphics[width=0.50\textwidth]{quicksort9.png}
				
				\footnote{http://www.algolist.net/Algorithms/Sorting/Quicksort}
				
				\vskip1cm
						
						\item En el mejor caso, el pivote termina en el centro de la lista, dividi�ndola en dos sublistas de igual tama�o. En este caso, el orden de complejidad del algoritmo es $O\left(n$ log $n)$.
						\item En el peor caso, el pivote termina en un extremo de la lista. El orden de complejidad del algoritmo es entonces de $O\left(n^{2})$. El peor caso depender� de la implementaci�n del algoritmo, aunque habitualmente ocurre en listas que se encuentran ordenadas, o casi ordenadas.
				\end{enumerate}
							
			\vskip3cm
		
		\section {Conclusiones}
		
		\vskip0.5cm
		
			\begin{enumerate}
			
				\item Este algoritmo de ordenaci�n es un ejemplo claro de que el m�todo divide y vencer�s es efectivo cuando tienes cantidades grandes de datos por trabajar y necesitas ahorrar tiempo y recursos.
				
				\item En el m�todo de ordenaci�n quicksort es mejor tener el pivote en el medio del vector puesto que es mas f�cil de implementar y su c�digo es mas sencillo.
			
			\end{enumerate}
	
\end{document}