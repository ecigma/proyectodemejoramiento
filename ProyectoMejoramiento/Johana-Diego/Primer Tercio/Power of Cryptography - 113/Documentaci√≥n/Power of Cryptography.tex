\documentclass[11pt]{article}
\usepackage[latin1]{inputenc}
\usepackage[spanish]{babel}
\usepackage{amsmath}
\usepackage{amssymb}
\usepackage{amsfonts}
\usepackage{multicol}
\usepackage{enumerate}
\usepackage{pstricks,pst-grad,pst-plot,pst-coil}
\usepackage{graphicx}
\newpsobject{grilla}{psgrid}{subgriddiv=1,griddots=10,gridlabels=6pt}
\setlength{\textwidth}{18cm}
\setlength{\textheight}{23cm}
\setlength{\footskip}{1.5cm}
\oddsidemargin-1cm
\topmargin-1.5cm
\setlength{\parindent}{0pt}
\date{}
\renewcommand{\columnseprule}{0.05pt}
\begin{document}
	
	\begin{center}
	\textbf{Age Sort}\\
	
	\textbf{Johanna Beltran y Diego Trivi�o}\\
	
	2012
	\end{center}
	
	\vskip8cm
	
	\tableofcontents
	
	\vskip15cm
	
		\section {Introducci�n}
		
			\vskip0.5cm
		
			\textsl '\textbf{Age Sort}' es un problema de programaci�n el cual encontramos en el juez virtual UVA con el n�mero \textbf{113}. 
			
			Este documento busca mostrar una de las tantas soluciones desde el enfoque matem�tico teniendo en cuenta que el objetivo es realizar la implementaci�n de la soluci�n del problema en cualquier lenguaje de programaci�n con la ayuda de este documento.
			
			Este problema puede ser resuelto utilizando la estrategia de 'divide y vencer�s'.
			
			Esta estrategia es una t�cnica de dise�o de algoritmos la cual consiste en dividir de forma recurrente un problema en subproblemas m�s sencillos hasta que se encuentre un caso base. 
			
			\vskip1cm
			
			Esta t�cnica consta fundamentalmente de los siguientes pasos:
			
			\begin{enumerate}
			
				\item Descomponer el problema a resolver en un cierto n�mero de subproblemas m�s peque�os.
				
				\item Resolver independientemente cada subproblema.
				
				\item Combinar los resultados obtenidos para construir la soluci�n del problema original.
				
			\end{enumerate}
		
		\vskip4cm
		
		\section {Definici�n del problema}
		
			\vskip0.5cm
			
			Este problema implica el c�lculo eficiente de la ra�z entera de un conjunto de n�meros.
		
			\vskip0.5cm
		
			Se deben tener en cuenta las siguientes restricciones para la soluci�n del problema:
			
			\begin{enumerate}
			
				\item Para cada caso de prueba se debe ingresar un n�mero $n \geq 1$ y un n�mero $p \geq 1$
				
				\item Para todos los pares de n�meros ingresados se tiene en cuenta que $1 \leq n \leq 200$ y $1 \leq p \leq 10^{101}$ .
				
			\end{enumerate}
	 
	 	\vskip7cm
	
				\subsection {Entrada}
				
				\vskip0.5cm
				
					La entrada consiste en una secuencia de pares de n�meros enteros $n$ y $p$ con cada n�mero entero en una l�nea diferente. Para todos los pares $1 \leq n \leq 200$ , $1 \leq p \leq 10^{101}$ existe un entero $k$, $1 \leq k \leq 10^{9}$ tal que $k^{n} = p$.
					
					\vskip0.5cm
					
					\[
					 	\emph{EJEMPLO}
					\]
					\[
					 		2
					\]
					\[
					 		16
					\]
					\[
					 		3
					\]
					\[
					 		27
					\]
					\[
					 		7
					\]
					\[
					 		4357186184021382204544
					\]
				
				\vskip2cm
				
				\subsection {Salida}
				
					\vskip0.5cm
				
					Para cada par de enteros $n$ y $p$ el valor $\sqrt[n]{p}$ debe ser impreso, es decir, el n�mero $k$ de tal manera que $k^{n} = p$.
					
					\vskip0.5cm
					
					\[
					 	\emph{EJEMPLO ANTERIOR}
					\]
					\[
							4
					\]
					\[
							3
					\]
					\[
							1234
					\]
	
			\vskip2cm
			
		\section {Modelamiento matem�tico}
		
		\vskip0.5cm
			
		Dados dos n�meros enteros positivos $n$ y $p$ se debe plantear un $k$ tal que $k^{n} = p$. 
					
				\vskip4cm
					
		\section {Planteamiento de la soluci�n}
		
		\vskip0.5cm
		
			Para cada caso de prueba se tendr� un n�mero $k$ que var�a entre $1 \leq k \leq \left(p\div n)$ hasta que $k^{n} = p$ en el cual se utilizar� una funci�n recursiva para calcular los exponentes de un n�mero.
			
				\vskip2cm
			
			La funci�n recursiva para calcular los exponentes de un n�mero ser� la siguiente:
				
					\[
							expo(p,n) = \begin{cases}
							\ $1$ ,& \text{si} \hskip0.3cm $n$ = 0
							\\ \text{expo} $(k,n/2) \times$ expo $(k$,$n/2)$ ,& \text{si} \hskip0.3cm $n$ \text{ es par}
							\\\ \text{expo} $ k \times (k,n-1)$ ,& \text{si} \hskip0.3cm $n$ \text{ es impar}
							\end{cases}
					\]
							
			\vskip3cm
		
		\section {Conclusiones}
		
		\vskip0.5cm
		
			\begin{enumerate}
			
				\item Por las caracter�sticas de este problema se recomienda utilizar el enfoque de la estrategia de 'divide y vencer�s' puesto que cada problema se reduce a un �nico subproblema m�s simple que un algoritmo general.
				
				\item Para este problema utilizamos una funci�n recursiva para calcular los exponentes de un n�mero que torna el problema a ser m�s eficiente, sin embargo,  se pueden implementar algoritmos no recursivos que almacenen las soluciones parciales en una estructura de datos expl�cita, como puede ser una pila o cola.
			
			\end{enumerate}
	
\end{document}