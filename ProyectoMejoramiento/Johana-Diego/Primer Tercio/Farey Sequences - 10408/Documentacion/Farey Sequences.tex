\documentclass[11pt]{article}
\usepackage[latin1]{inputenc}
\usepackage[spanish]{babel}
\usepackage{amsmath}
\usepackage{amssymb}
\usepackage{amsfonts}
\usepackage{multicol}
\usepackage{enumerate}
\usepackage{pstricks,pst-grad,pst-plot,pst-coil}
\usepackage{graphicx}
\newpsobject{grilla}{psgrid}{subgriddiv=1,griddots=10,gridlabels=6pt}
\setlength{\textwidth}{18cm}
\setlength{\textheight}{23cm}
\setlength{\footskip}{1.5cm}
\oddsidemargin-1cm
\topmargin-1.5cm
\setlength{\parindent}{0pt}
\date{}
\renewcommand{\columnseprule}{0.05pt}
\begin{document}
	
	\begin{center}
	\textbf{Farey Sequences}\\
	
	\textbf{Johanna Beltran y Diego Trivi�o}\\
	
	2012
	\end{center}
	
	\vskip8cm
	
	\tableofcontents
	
	\vskip15cm
	
		\section {Introducci�n}
		
			\vskip0.5cm
		
			\textsl '\textbf{Farey Sequences}' es un problema de programaci�n el cual encontramos en el juez virtual UVA con el n�mero \textbf{10408}. 
			
			Este documento busca mostrar una de las tantas soluciones desde el enfoque matem�tico teniendo en cuenta que el objetivo es realizar la implementaci�n de la soluci�n del problema en cualquier lenguaje de programaci�n con la ayuda de este documento.
			
			Este problema puede ser resuelto utilizando la estrategia de 'divide y vencer�s'.
			
			Esta estrategia es una t�cnica de dise�o de algoritmos la cual consiste en dividir de forma recurrente un problema en subproblemas m�s sencillos hasta que se encuentre un caso base. 
			
			\vskip1cm
			
			Esta t�cnica consta fundamentalmente de los siguientes pasos:
			
			\begin{enumerate}
			
				\item Descomponer el problema a resolver en un cierto n�mero de subproblemas m�s peque�os.
				
				\item Resolver independientemente cada subproblema.
				
				\item Combinar los resultados obtenidos para construir la soluci�n del problema original.
				
			\end{enumerate}
		
		\vskip3cm
		
		\section {Definici�n del problema}
		
			\vskip0.5cm
			
			Para este problema se tiene en cuenta la \textbf{secuencia de Farey} de orden $n$, $F_{n}$ , es la secuencia de todas las fracciones propias con denominadores que no superen $n$  junto con la fracci�n $1/1$ y ordenadas ascendentemente; tanto el numerador como el denominador no deben tener factores comunes.
			
			Por ejemplo, la secuencia de $F_{5}$ es:
			
			\vskip1cm
			
			\includegraphics[width=0.90\textwidth]{farey1.png}
			
			\vskip0.5cm
			
			La soluci�n del problema se encuentra en que para un determinado $n$, se debe encontrar la $k -�sima$ fracci�n de la secuencia $F_{n}$.
		
			\vskip0.5cm
		
			Se deben tener en cuenta las siguientes restricciones para la soluci�n del problema:
			
			\begin{enumerate}
			
				\item Para cada caso de prueba se debe ingresar dos n�meros enteros $n$ y $k$.
				
				\item Para $n$ y $k$ se debe tener en cuenta que $1 \leq n \leq 1000$ y $k$ debe ser mas peque�o que la secuencia obtenida de $F_{n}$; la longitud de $F_{n}$ es de aproximadamente 0,3039635 $n^{2}$.
				
			\end{enumerate}
	 
	 	\vskip7cm
	
				\subsection {Entrada}
				
				\vskip0.5cm
				
					La entrada consiste en una secuencia de l�neas que contienen dos n�meros naturales $n$ y $k$ , $1 \leq n \leq 1000$ y $k$ suficientemente peque�o tal que no es el $k -�simo$ t�rmino en $F_{n}$ . (La longitud de $F_{n}$ es de aproximadamente 0,3039635 $n^{2}$ ).
					
					\vskip0.5cm
					
					\[
					 	\emph{EJEMPLO}
					\]
					\[
					 		5 \begin{tabular}{ccc} 5 \end {tabular}
					\]
					\[
					 		5 \begin{tabular}{ccc} 1 \end {tabular}
					\]
					\[
					 		5 \begin{tabular}{ccc} 9 \end {tabular}
					\]
					\[
					 		5 \begin{tabular}{ccc} 10 \end {tabular}
					\]
					\[
					 		117 \begin{tabular}{ccc} 348 \end {tabular}
					\]
					\[
					 		288 \begin{tabular}{ccc} 10000 \end {tabular}
					\]
				
				\vskip2cm
				
				\subsection {Salida}
				
					\vskip0.5cm
				
					Para cada caso de prueba se debe imprimir una l�nea con la $k -�sima$ fracci�n de $F_{n}$.
					
					\vskip0.5cm
					
					\[
					 	\emph{EJEMPLO ANTERIOR}
					\]
					\[
							1/2
					\]
					\[
							1/5
					\]
					\[
							4/5
					\]
					\[
							1/1
					\]
					\[
							9/109
					\]
					\[
							78/197
					\]
	
			\vskip2cm
			
		\section {Modelamiento matem�tico}
		
		\vskip0.5cm
			
		Dado dos n�meros enteros positivos $n$ y $k$ se debe encontrar la fracci�n $k-esima$ de la secuencia ordenada $F_{n}$.
					
				\vskip4cm
					
		\section {Planteamiento de la soluci�n}
		
		\vskip0.5cm
		
			Para este problema utilizaremos el algoritmo de ordenamiento \textbf{Mergesort} tambien llamado algoritmos de ordenamiento por mezcla el cual consiste en el ordenamiento externo estable basado en la tecnica de divide y venceras;  el algoritmo trabaja de la siguiente forma:
			
				\vskip1cm

				\includegraphics[width=0.50\textwidth]{mergesort1.png}
				
				\footnote{http://throwingcodes.blogspot.com/}
				
				\vskip1cm
				
				\begin{enumerate}
					\item Si la longitud de la lista es 0 � 1, entonces ya est� ordenada.
					\item Dividir la lista desordenada en dos sublistas de aproximadamente la mitad del tama�o.
					\item Ordenar cada sublista recursivamente aplicando el ordenamiento por mezcla.
					\item Mezclar las dos sublistas en una sola lista ordenada.
				\end{enumerate}
								
				Con este algoritmo de ordenamiento podemos ordenar de manera ascendente las fracciones de la secuencia $F_{n}$ para asi determinar la $k-esima$ fracci�n de cada caso de prueba dada.
			
			\vskip1cm
				
			\vskip5cm
		
		\section {Conclusiones}
		
		\vskip0.5cm
		
			\begin{enumerate}
			
				\item Este algoritmo de ordenaci�n es un ejemplo claro de que el m�todo divide y vencer�s es efectivo cuando tienes cantidades grandes de datos por trabajar y necesitas ahorrar tiempo y recursos.
				
				\item Este algoritmo de ordenamiento tiene una complejidad de $\left( n$ log $n)$ y requiere de un vector auxiliar para realizar el ordenamiento correctamente.
			
			\end{enumerate}
	
\end{document}